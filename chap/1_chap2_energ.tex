\section{Conducteurs isolés}

\noindent\textbf{Définition} : L'équilibre électrostatique d'un conducteur est atteint lorsqu'une charge électrique ne se déplace plus à l'intérieur du conducteur. Cela entraine nécessairement que :
\begin{itemize}
\item Le champ électrostatique total est nul dans le conducteur. (i.e. $\vec{E}=0$ et donc $V=\textrm{cte}$)
\item Les charges sont localisées à la surface.\\
\end{itemize}
\noindent\textbf{Théorème} : Le champ électrostatique à proximité immédiate d'un conducteur de densité surfacique $\sigma$ vaut :
\[ \vec{E}=\frac{\sigma}{\varepsilon_0}\vec{n} \]

\section{Influence électrostatique}
\quadri{20}

\section{Application : Condensateur}
\quadri{30}
