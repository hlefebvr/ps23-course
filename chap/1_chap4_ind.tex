\section{Electrocinétique}
\subsection{Loi d'Ohm}

\emph{Rappel} : \quadri{5} Le courant traversant un tronçon de circuit de section $S$ est \[ I=\frac{dQ}{dt}=\iint_S\vec{j}.\vec{n}dS \]
Dans la plupart des conducteurs, on observe une proportionalité entre la densité de courant ($\vec{j}$) et le champ électromoteur : $\vec{E}_m=\frac{\vec{F}}{q}$ (où $\vec{F}$ est la force responsable du mouvement des électrons)\\\\
\noindent\textbf{Loi d'ohm} : \[ \vec{j}=\gamma\vec{E}_m \]
Où $\gamma$ est appelé conductivité et $\eta=\gamma^{-1}$ est appelé la résistivité.\\\\
\noindent On défnit alors la résistance d'un conducteur de section $S$ et de longueur $L$ par
\[ R=\frac{\int_L\vec{E}_m.\vec{dl}}{I=\iint_S\vec{j}.\vec{n}dS}=\frac{\int_L\vec{E}_m.\vec{dl}}{\iint_S\gamma\vec{E}_m.\vec{n}dS} \]
\emph{Remarque} : Si $E_m$ est constant, alors $R=\frac{\eta L}{S}$\\\\
\noindent\emph{Remarque 2} : Si  le champ électromoteur est d'origine électrostatique pure, alors $\vec{E}_m=-\vec{\textrm{grad }}V$ et donc : 
\[ R=\frac{\int_L\vec{E}_m.\vec{dl}}{I}=\frac{-\int_L\vec{\textrm{grad }}V.\vec{dl}}{I}=\frac{V_A-V_B}{I} \]

\subsection{Conservation de la charge}
Considérons une surface $S$ délimité par un volume $\nu$. Alors,
\[ \frac{dQ}{dt}=\iiint_\nu\frac{\partial \rho}{\partial \rho}d\nu=-\iint_S\vec{j}.\vec{n}dS=\iint_\nu\textrm{div }\vec{j}d\nu \]
D'où finalement : 
\[ \frac{\partial\rho}{\partial t}+\textrm{div }\vec{j}=0 \]

\section{Induction}
\subsection{Expérience de Faraday}
\quadri{13}
\noindent\textbf{Loi de Faraday} : La variation temporelle du flux magnétique à travers un circuit fermé y engendre une force électromotrice induite notée $e$ ou $fem$ telle que
\[ e=\oint_C\vec{E}_m.\vec{dl}=-\frac{d\Phi}{dt} \]
Où $\Phi=\iint_S\vec{B}.\vec{n}dS$\\\\
\noindent\textbf{Loi de Lenz} : L'induction produit des effets qui s'opposent aux causes qui lui ont donné naissance.\\
Par exemple, Si $B$ augmente, le circuit $C$  va être telle qu'il va créer un champ $B_2$ qui va s'opposer à l'accroissement de $B$.
\subsection{Equation de Maxwell-Faraday}
Lorsqu'un courant variable circule dans un circuit, un champ magnétique est créé. Dans ce cas précis, la force induite doit être dû à la composante électrique de la force de Lorentz. On a :
\[ e=\oint_C\vec{E}.\vec{dl}=-\frac{d}{dt}\left(\iint_S\vec{B}.\vec{n}dS\right) \]
Or, on a (MT22) :
\[ \oint_C\vec{E}.\vec{dl}=\iint_S\vec{\textrm{rot }}\vec{E}.\vec{n}dl \]
D'où enfin, l'équation  de Maxwell-Faraday
\[ \vec{\textrm{rot }}\vec{E}=-\frac{\partial\vec{B}}{\partial t} \]

\subsection{Auto-induction et induction mutuelle}

Un circuit isolé est parcouru par un courant $I$, ce courant engendre un champ magnétique, et donc un flux
\[ \Phi=\iint_S\vec{B}.\vec{n}dS=\left[\frac{\mu_0}{4\pi}\iint_S\left(\oint_C\frac{\vec{dl}\land\vec{u}}{r^2}\right)\right].I=L.I \]
En notant
\[ L= \frac{\mu_0}{4\pi}\iint_S\left(\oint_C\frac{\vec{dl}\land\vec{u}}{r^2}\right) \]
Enfin, d'après Faraday, la relation bien connue suivante est toujours vérifiée :
\[ e=-L\frac{dI}{dt} \]

\subsection{Retour sur les équations de Maxwell}

\quadri{30}

